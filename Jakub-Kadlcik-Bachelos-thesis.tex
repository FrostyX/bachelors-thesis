\documentclass[10pt,a4paper]{article}
\usepackage[utf8]{inputenc}
\usepackage[czech]{babel}
\usepackage[T1]{fontenc}

\title{Determine applications affected by upgrade}
\author{Jakub Kadlčík}

\begin{document}
	% Vazba
	\maketitle
	\newpage
	
	% Anotace
	
	% Poděkování
	
	\tableofcontents
	\newpage
	
	\section{Úvod}
		\subsection{Správa software v GNU/Linux}
		Ve většině distribucí GNU/Linuxu se software standardně spravuje prostřednictvím balíčkovacího systému. Nové distribuce většinou vznikají odvozením jiné, již existující distribuce a v tomto případě bývá zachován typ balíčků i balíčkovací systém. Můžeme tedy říct, že distribucí je mnoho, ale balíčkovaích systémů se používá jen velmi málo. Konkrétně lze hovořit o distribucích GNU/Linuxu používající balíčky RPM\footnote{RPM = Red Hat Package Manager}, DEB\footnote{DEB = Debian software package} a nebo jiný typ balíčku. V posledním případě se jedná případ od případu o velmi specifický předpis balíčku. V rámci této práce se budeme zabývat linuxovou distribucí Fedora jako reprezentantem první skupiny a distribucí Gentoo, jako reprezentantem poslední skupiny.
		
		\subsection{Balíčkovací systém distribuce Fedora}
		Fedora vznikla jako nekomerční odnož Red Hat Linuxu a zachovala stávající systém balíčků RPM. Od své první verze využívá nástavbu nad RPM zvanou YUM. Ten se po čase stal téměř neudržovatelný a proto se objevila alternativa pojmenovaná DNF. V současné době lze paralelně využívat oba tyto nástroje, avšak je snaha YUM kompletně nahradit za pomocí DNF. Mělo by k tomu dojít již ve Fedoře 21. Aktuální stabilní verze je 20, jde tedy o příští vydání.
				
		\subsection{Balíčkovací systém distribuce Gentoo}

	\section{Problém}
	Když spustíme aplikaci, do paměti se načtou knihovny a soubory potřebné k jejímu běhu. Pokud je některá z knihoven, nebo samotná aplikace po dobu jejího běhu aktualizována, v paměti stále zůstanou původní soubory, přestože fyzicky už nemusí existovat (mohou být smazány, nebo nahrazeny novějšími verzemi).
	\\\\
	Zjistit, které, v systému spuštěné, aplikace využívají takto neaktuální soubory, není pro uživatele triviální. V současné době totiž neexistuje dostatečný nástroj, který by tento problém řešil.
	
	\section{Požadavky}
	Tato práce si klade za cíl vytvořit aplikaci, která bude řešit dříve zmíněný problém. Aplikace by měla najít všechny programy využívající neaktuální soubory a doporučit jejich restartování. Pro různé typy programů by měla být vypsána jiná doporučení a nápověda, jakým způsobem to lze provést. Aplikace by měla disponovat textovým uživatelským rozhraním a mělo by být možné ji používat jako modul přímo v balíčkovacím systému DNF. Vývoj by měl probíhat pod otevřenou licencí a na libovolném veřejném úložišti.
	
	\section{Analýza}
	\section{Implementace}
		\subsection{Testování}

	\section{Integrace do operačních systémů}
		\subsection{Fedora}
		\subsection{Gentoo}
		
	\section{Závěr} % Zrušit číslování
	\section{Conclusions} % Zrušit číslování
	\section{Reference} % Zrušit číslování
	
	% Obsah přiloženého CD
\end{document}
